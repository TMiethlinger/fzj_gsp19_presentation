\part{Examples}
\makepart
\section{Features}
\selectlanguage{english}
\begin{frame}
        \frametitle{\LaTeX{}-Beamer Features}
        The following slides show how {\tt Latex-Beamer} constructs work within the
        template.
        \begin{itemize}
      %\item Framebreaks
      \item Lists, numbered lists
      \item Plain slides, background images
      \item Theorems, proofs
      \item Definitions, examples
      \item Blocks, alert blocks
      \item Highlight options
      \item Formulae
      \item Verbatim environments
    \end{itemize}
\end{frame}

\section{Lists}
\begin{frame}
        \frametitle{Lots of lists}
        \framesubtitle{Another Subtitle}
        \begin{itemize}
          \item using the \texttt{pause} command:
          \begin{itemize}
            \item First item.
            \pause
            \item Second item.
          \end{itemize}
          \item using overlay specifications:
          \begin{enumerate}
            \item<3-> First numbered item.
            \item<4-> Second numbered item.
            \begin{itemize}
              \item 3rd level item!
            \end{itemize}
          \end{enumerate}
          \item using the general \texttt{uncover} command:
          \begin{itemize}
            \uncover<5->{\item First item.}
            \uncover<6->{\item Second item.}
          \end{itemize}
        \end{itemize}
\end{frame}

\begin{frame}[fragile]
        \frametitle{Plain Frames}
        \begin{itemize}
      \item The next slide shows a plain frame.
      \item To use plain frames add the \verb![plain]! parameter to your \verb!\begin{frame}! statement.
    \end{itemize}
        \begin{block}{How to use plain frames}
    \scriptsize
    \verbatiminput{frame_plain}
        \end{block}
\end{frame}

%\begin{frame}[plain]
    \frametitle{Plain Frame}
    \begin{center}
        Here is my tiny text on a plain frame.
    \end{center}
\end{frame}


\begin{frame}[c,plain]
        \frametitle{Plain Frame}
        \begin{center}
                {\tiny Enough} {\scriptsize space} for {\Large your} {\huge big} {\Huge ideas.} {\TINY (or holiday pictures)}
        \end{center}
\end{frame}

\section{Backgrounds}
\begin{frame}[fragile]
    \frametitle{Background Images}
    \framesubtitle{On Standard Frames}
    \begin{itemize}
      \item The next slide shows an image, embedded into the background of the frame layout.
      \item The background image is automatically cropped to the frame dimensions.
    \end{itemize}
    \begin{block}{How to install a background image}
        \scriptsize
        \verbatiminput{frame_background}
    \end{block}
\end{frame}

\setbeamertemplate{background}{\includegraphics[width=\paperwidth]{placeholder}}
\begin{frame}
    \frametitle{An image in the background}
    \centering
    Some text in front of the background image.
\end{frame}
\setbeamertemplate{background}{}


\section{Beamer Block Constructs}
\subsection{Theorem, Proof}
\begin{frame}
        \frametitle{Block Constructs}
        \framesubtitle{{\tt theorem, proof}}
        \begin{theorem}
        There is no largest prime number.
        \end{theorem}

        \begin{proof}
                \begin{enumerate}
                        \item<1-| alert@1> Suppose $p$ were the largest prime number.
                        \item<2-> Let $q$ be the product of the first $p$ numbers.
                        \item<3-> Then $q+1$ is not divisible by any of them.
                        \item<1-> Thus $q+1$ is also prime and greater than $p$.\qedhere
                \end{enumerate}
        \end{proof}
\end{frame}

\subsection{Definition, Example}
\begin{frame}
        \frametitle{Block Constructs}
        \framesubtitle{{\tt definition, example}}
        \begin{definition}
                A \alert{prime number} is a number that has exactly two divisors.
        \end{definition}
        \begin{example}
                \begin{itemize}
                        \item 2 is prime (two divisors: 1 and 2).
                        \item 3 is prime (two divisors: 1 and 3).
                        \item 4 is not prime (\alert{three} divisors: 1, 2, and 4).
                \end{itemize}
        \end{example}
\end{frame}

\subsection{Block, Alert Block}
\begin{frame}
        \frametitle{Block Constructs}
        \framesubtitle{{\tt block, alertblock}}
        \begin{block}{Simple Block}
                Just some text.
        \end{block}
        \begin{alertblock}{Alert Block}
                This block seems to be pretty important.
        \end{alertblock}
\end{frame}

\section{Highlight important information}
\begin{frame}[fragile]
        \frametitle{Highlight important information}
        \framesubtitle{Use ``Jülich'' colors to attract attention}
        \begin{block}{Use {\tt \textbackslash{}emph\{\}}}
                \verb+This text is \emph{important}.+ \\
                This text is \emph{important}.
        \end{block}
        \begin{block}{Use {\tt \textbackslash{}alert\{\}}}
                \verb+This text is \alert{really} important!+ \\
                This text is \alert{really} important!
        \end{block}
\end{frame}

\section{Math Environment}
\begin{frame}
        \frametitle{Math Environment}
        \framesubtitle{Use your {\LaTeX} formulae inside your slides without hassle}
        \[
            \iiint\limits_V \operatorname{div} \vec{F} \, dV
            = \iint\limits_S \vec{F}\cdot d\vec{S}
        \]
        \[
         \prod_{k=1}^n k = n! \,,\quad \sum_{k=1}^n k=\frac{n(n+1)}{2}\,,
          \quad \int_0^{2\pi}\sin t\,dt=0
        \]
        \[
            p(x)=\sum_{i=0}^n f_{i}q_{i}(x) \quad\mbox{with}\quad
            q_{i}(x)=\prod_{\substack{k=0 \\ k\neq i}}^n
            \frac{x-x_{k}}{x_{i}-x_{k}}\,.
        \]
        \[
            \iint\limits_S (U \operatorname{grad} W)\cdot d\vec{S}
            =\iiint\limits_V (\operatorname{grad} U\cdot
             \operatorname{grad} W +U\Delta W)\,dV
        \]
\end{frame}

\section{Code Environment}
\begin{frame}[fragile]
        \frametitle{Verbatim Environment}
        \framesubtitle{Code Snippets}
        \begin{itemize}
      \item Slides containing \verb!\verb! statements must be defined \verb+fragile+
    \end{itemize}
    \scriptsize
    \verbatiminput{frame_verbatim}
\end{frame}

\input{frame_verbatim}
