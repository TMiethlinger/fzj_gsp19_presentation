\part{Jülich Colors}
\makepart

\begin{frame}[label=colors]
  \frametitle{Corporate Colors}
  \framesubtitle{You can use predefined colornames to spice up your slides}
  \centering
  \begin{tikzpicture}
    \foreach \color [count=\i] in {fzjorange, fzjviolet, fzjyellow, fzjgreen, fzjred, fzjlightblue, fzjblue} {
        \node[fill=\color,circle,minimum size=2.3cm] at (-\i*360/7+90: 2.3cm) {\color};
    }
  \end{tikzpicture}
\end{frame}

\begin{frame}[fragile]
    \frametitle{Using Corporate Colors}
    In text:
    \begin{itemize}
        \item \verb+\textcolor{colorname-text}{text}+\\
            There is a \textcolor{fzjgreen}{green} word in this sentence.
        \item \verb+\colorbox{colorname-background}{content}+\\
            \colorbox{fzjorange}{This text is on an \textcolor{fzjred}{orange} background.}
        \item \verb+\fcolorbox{colorname-frame}{colorname-background}{content}+\\
            \fcolorbox{fzjblue}{fzjlightblue}{\textcolor{fzjviolet}{This colored text is in a colorful framed box.}}
    \end{itemize}

    In TikZ, pgfplots: use the named colors in any color specification.
\end{frame}
