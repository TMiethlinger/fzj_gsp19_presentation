\begin{frame}
\frametitle{Prebuilt Slides A1}
\begin{block}{Why?}
  \begin{itemize}
    \item Speed up compilation time
    \item Rebuild expensive slides with TikZ pictures or pgfplots plots only after they changed
  \end{itemize}
\end{block}
\end{frame}

\begin{frame}
\frametitle{Prebuilt Slides II}
\begin{block}{How?}
  \begin{itemize}
    \item Split presentation into multiple \texttt{Slides/*.tex} files
    \item Each contains just \texttt{\textbackslash begin\{frame\}...\textbackslash end\{frame\}} blocks
    \item A separate \texttt{.pdf} file is created for each \texttt{.tex} file automatically
    \item Three new commands to include these slides from the main document
      \begin{itemize}
        \item \texttt{\textbackslash includetitle\{\}}
	\item \texttt{\textbackslash includeframeset\{\}}
	\item \texttt{\textbackslash includeframeoverlay\{\}}
      \end{itemize}
    \item Footer will be added while including them in the main document
  \end{itemize}
\end{block}
\end{frame}

\begin{frame}
\frametitle{Prebuilt Slides III}
\begin{block}{Caveats}
  Not all \LaTeX{} features can be used with prebuilt slides:
  \begin{itemize}
    \item \texttt{\textbackslash ref\{\}} references to parts/sections/frames
    \item \texttt{\textbackslash cite\{\}} citations --
      But is a reference like \emph{[4]} really helpful in a presentation?\\
      Maybe just a manual reference like \emph{(Müller-Lüdenscheidt, 1978)} is better.
  \end{itemize}
\end{block}
\end{frame}

\begin{frame}
\frametitle{This is a frameset example}
\framesubtitle{First slide}
\begin{itemize}
  \item By using a frameset, each pdf page gets its own slide number
  \item You may put arbitrary many consecutive frames in a single .tex-file
  \item Framesets are included via \texttt{\textbackslash includeframeset\{\}}
\end{itemize}
\end{frame}

\begin{frame}
\frametitle{This is a frameset example}
\framesubtitle{Second slide}
\begin{enumerate}
  \item First bullet
  \item Second bullet
\end{enumerate}
\end{frame}
