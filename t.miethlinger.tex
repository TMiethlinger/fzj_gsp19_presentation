% Do not edit. These are fallback values, the actual values may have been
% set by the build system elsewhere.
\providecommand\topdir{.}%
\providecommand\handoutoption{}% option, normal beamer presentation mode by default


% Optional parameters for the documentclass should be the only
% things to be changed in this file.
\documentclass[
%%%%%%%%%% Start of customizable section %%%%%%%%%%
english,
ngerman,
10pt,%             supported values: 10pt, 11pt, 12pt
aspectratio=169,%  common values: 169, 1610, 43
%%%%%%%%%% End of customizable section %%%%%%%%%%
\handoutoption
]{beamer}


\usepackage{etoolbox}
\newtoggle{prebuiltslides}
\ifdefined\prebuiltslides
  \toggletrue{prebuiltslides}
\else
  \togglefalse{prebuiltslides}
\fi

% do not change the default packages
% allow many more packages, keep this as first package
\usepackage{etex}
\usepackage{luacode}
\usepackage{luatex85}

\usepackage[ngerman,main=english]{babel}

% Do not change this file

%define colors
\usepackage{xcolor}

\usepackage{graphics}

% change document reference style
\usepackage{hyperref}

\lstset{ language=C++,
         basicstyle=\footnotesize\ttfamily, % Standardschrift
         numbers=none,               % Ort der Zeilennummern
         numberstyle=\footnotesize,          % Stil der Zeilennummern
         %stepnumber=2,               % Abstand zwischen den Zeilennummern
         numbersep=5pt,              % Abstand der Nummern zum Text
         tabsize=2,                  % Groesse von Tabs
         extendedchars=true,         %
         breaklines=true,            % Zeilen werden Umgebrochen
         keywordstyle=\color{fzjblue},
         frame=,
         keywordstyle=[1]\textbf,    % Stil der Keywords
         keywordstyle=[2]\textbf,    %
         keywordstyle=[3]\textbf,    %
         keywordstyle=[4]\textbf,   %\sqrt{\sqrt{}} %
         rulesepcolor=\color{fzjblue},
         fillcolor=\color{fzjblue},
         stringstyle=\color{red!50!brown}\ttfamily, % Farbe der String
         showspaces=false,           % Leerzeichen anzeigen ?
         showtabs=false,             % Tabs anzeigen ?
         xleftmargin=15pt,
         framexleftmargin=1pt,
         framexrightmargin=1pt,
         framexbottommargin=1pt,
         %backgroundcolor=\color{fzjblue},
         showstringspaces=false,      % Leerzeichen in Strings anzeigen ?
         nolol=true,
 }

\lstdefinelanguage
   {x64-Assembler}     % add a "x64" dialect of Assembler
   [x86masm]{Assembler} % based on the "x86masm" dialect
   % with these extra keywords:
   {morekeywords={vsubpd,vmulpd,vaddpd,vsqrtpd,vdivpd,
                  vsubps,vmulps,vaddps,vsqrtps,vdivps,
                  vmovss,vsubss,vmulss,vaddss,vsqrtss,vdivss,
                  vmovsd,vsubsd,vmulsd,vaddsd,vsqrtsd,vdivsd,
                  vmovaps,vmovapd,
                  vfmadd231ps,vfmadd231pd,
                  xmm,ymm,
                  CDQE,CQO,CMPSQ,CMPXCHG16B,JRCXZ,LODSQ,MOVSXD, %
                  POPFQ,PUSHFQ,SCASQ,STOSQ,IRETQ,RDTSCP,SWAPGS, %
                  rax,rdx,rcx,rbx,rsi,rdi,rsp,rbp, %
                  r8,r8d,r8w,r8b,r9,r9d,r9w,r9b}}
 \lstloadlanguages{% Check Dokumentation for further languages ...
         x64-Assembler,
         C,
         C++,
 }
\lstset{language=C++,
    keywordstyle=\color{fzjblue}\bfseries,
    commentstyle=\color{green!60!black},
    stringstyle=\ttfamily\color{red!50!brown},
    morekeywords={Real,Scalar}
    }
\lstset{literate=%
   *{0}{{{\color{red!20!violet}0}}}1
    {1}{{{\color{red!20!violet}1}}}1
    {2}{{{\color{red!20!violet}2}}}1
    {3}{{{\color{red!20!violet}3}}}1
    {4}{{{\color{red!20!violet}4}}}1
    {5}{{{\color{red!20!violet}5}}}1
    {6}{{{\color{red!20!violet}6}}}1
    {7}{{{\color{red!20!violet}7}}}1
    {8}{{{\color{red!20!violet}8}}}1
    {9}{{{\color{red!20!violet}9}}}1
}

%\DeclareCaptionFont{fzjblue}{\color{fzjblue}}
%\DeclareCaptionFont{fzjgray}{\color{fzjgray}}
%%  \captionsetup[lstlisting]{singlelinecheck=false, labelfont={fzjblue}, textfont={fzjblue}}


%\DeclareCaptionFont{white}{\color{white}}
%\DeclareCaptionFormat{listing}{\colorbox[cmyk]{0.09, 0.05, 0.05,0.01}{\parbox[b][1.5ex][c]{\textwidth}{\hspace{15pt}#1#2#3}}}
%%\captionsetup[lstlisting]{format=listing,labelfont=fzjblue,textfont=fzjgray, singlelinecheck=false, margin=0pt, font={sc, bf,footnotesize}}
%\captionsetup{labelformat=empty,labelsep=none}

%pgf tikz and related
\usepackage{pgf}
\usepackage{tikz}
\usepackage{fp}
\usepackage{pgfplots}

%allow environment with tikz
\usepackage{environ}

\usepackage{tikz-3dplot}

% settings for the design of tables
\arrayrulecolor{fzjblue}

% allow watermark from git hash key
\iftoggle{prebuiltslides}{
  % no git info for prebuilt slides
}{
  \usepackage{gitinfo2}
}



% add required packages inside the packages.local.tex
% additional packages you need to include should go here


% do not change the default config
% Do not change this file, all changes must go into the config.local.tex file

% prefixes the includegraphics path with these directories
\graphicspath{{Images/}{Plots/}{Figures/}}

% external config files for different packages and/or features
\newcommand{\modesuffix}{.beamer}

%beamer handout definitions
\mode<handout>{
  \usetheme{Juelich}
  \iftoggle{prebuiltslides}{
    \pgfpageslogicalpageoptions{1}{border code=}
    \pgfpagesuselayout{resize to}[
      physical paper height=\paperheight,
      physical paper width=\paperwidth]
  }{
  \pgfpagesuselayout{2 on 1}[a4paper,border shrink=5mm]
  }
  \renewcommand{\modesuffix}{.handout}
}

% use the Juelich theme for all slides in presentation mode
\mode<presentation>{
  \usetheme{Juelich}
}

%\fzjset{title=allcaps}         % to set the title in allcaps
 \fzjset{title=regular}         % to set the title regular
%\fzjset{subtitle=allcaps}      % to set the title in allcaps for short text
%\fzjset{subtitle=regular}      % to set the title regular and in a smaller font for long text
%\fzjset{part=allcaps}          % to set the part in allcaps for short text
%\fzjset{part=regular}          % to set the part regular and in a smaller font for long text
%\fzjset{frametitle=allcaps}    % to set the frametitle in allcaps for short text
 \fzjset{frametitle=regular}    % to set the frametitle regular font for long text

% define \emph as color version, no bold, slanted, italic, please
\renewcommand{\emph}[1]{\structure{#1}}

% simply incremental slides inside tikz
\tikzset{
  onslide/.code args={<#1>#2}{%
    \only<#1>{\pgfkeysalso{#2}} % \pgfkeysalso doesn't change the path
  }
}

\DeclareRobustCommand{\includetitle}[1]{
  \pdfximage{#1\modesuffix.pdf}
  \pgfmathparse{\the\pdflastximagepages}
  \foreach \x in {1,...,\pgfmathresult} {
  \setbeamertemplate{frame number}[invisible]%
  \setbeamertemplate{date}[invisible]%
  \setbeamertemplate{background canvas}{
    \includegraphics[page=\x,width=\paperwidth,height=\paperheight,keepaspectratio]{#1\modesuffix}}
  \begin{frame}\end{frame} \setbeamertemplate{background canvas}{}
  \addtocounter{framenumber}{-1} % counter will increase on every frame
  }
}

\DeclareRobustCommand{\includeframeset}[1]{
  \pdfximage{#1\modesuffix.pdf}
  \pgfmathparse{\the\pdflastximagepages}
  \foreach \x in {1,...,\pgfmathresult} {
  \setbeamertemplate{background canvas}{
    \includegraphics[page=\x,width=\paperwidth,height=\paperheight,keepaspectratio]{#1\modesuffix}}
  \begin{frame}\end{frame} \setbeamertemplate{background canvas}{}
%  \addtocounter{framenumber}{-1} % counter will increase on every frame
  }
}

\DeclareRobustCommand{\includeframeoverlay}[1]{
  \pdfximage{#1\modesuffix.pdf}
  \pgfmathparse{\the\pdflastximagepages}
  \foreach \x in {1,...,\pgfmathresult} {
  \setbeamertemplate{background canvas}{
    \includegraphics[page=\x,width=\paperwidth,height=\paperheight,keepaspectratio]{#1\modesuffix}}
  \begin{frame}\end{frame} \setbeamertemplate{background canvas}{}
  \addtocounter{framenumber}{-1} % keeps counter constant
  }
  \addtocounter{framenumber}{+1} % sets framecounter for next slide outside this group
}

% TODO copy in beamertheme-juelich template
\definecolor{fzjblack}{RGB}{0,0,0}

\usefonttheme{professionalfonts} % needed to use math symbols from fontenc

\usepackage{amsmath}
\usepackage{latexsym} % replaces amssymb

\usepackage{unicode-math} % enables \setmathfont commands
\usepackage{fontawesome} % loads fancy web symbols

\setmainfont[
  Scale=0.9,
  Extension=.ttf,
  UprightFont=*-Regular,
  BoldFont=*-Bold,
  ItalicFont=*-Italic,
  BoldItalicFont=*-BoldItalic,
  SlantedFont=*-Italic,
]{LiberationSans}

\setsansfont[
  Scale=0.9,
  Extension=.ttf,
  UprightFont=*-Regular,
  BoldFont=*-Bold,
  ItalicFont=*-Italic,
  BoldItalicFont=*-BoldItalic,
  SlantedFont=*-Italic,
]{LiberationSans}

\setmonofont[
  Scale=0.9,
  Extension=.ttf,
  UprightFont=*-Regular,
  BoldFont=*-Bold,
  ItalicFont=*-Italic,
  BoldItalicFont=*-BoldItalic,
  SlantedFont=*-Italic,
]{LiberationMono}

\unimathsetup{math-style=ISO} % french, ISO, TeX, upright

\setmathfont[Scale=1]{TeX Gyre Bonum Math}
\setmathfont[Scale=0.9,range=\mathup/{latin,Latin,num}]{LiberationSans}
\setmathfont[Scale=0.9,range=\mathit/{latin,Latin,num}]{LiberationSans-Italic}
\setmathfont[Scale=0.9,range=\mathbfup/{latin,Latin,num}]{LiberationSans-Bold}
\setmathfont[Scale=0.9,range=\mathbfit/{latin,Latin,num}]{LiberationSans-BoldItalic}
\setmathfont[Scale=1,range=\mathcal/{latin,Latin}]{jsMath-cmsy10} % loads  nice calligraph symbols
\setmathfont[Scale=1,range=\mathbb/{greek,Greek,latin,Latin,num}]{TeX Gyre Bonum Math} % loads  nice calligraph symbols

% change document reference style
\usepackage{hyperref}

%pgf tikz and related
\usepackage{pgf}
\usepackage{tikz}
\usepackage{fp}
\usepackage{pgfplots}

%allow environment with tikz
\usepackage{environ}

\usepackage{tikz-3dplot}

\pgfplotsset{compat=1.14}

\pgfplotscreateplotcyclelist{fzjblue}{
{fzjblue},
{fzjblue!80!white},
{fzjblue!70!white},
{fzjblue!60!white},
{fzjblue!50!white},
{fzjblue!40!white},
{fzjblue!30!white},
{fzjblue!20!white},
{fzjblue!10!white},
}

\pgfplotscreateplotcyclelist{fzjdiagram}{
{fzjblue},
{fzjorange},
{fzjgray!50!black},
{fzjviolet},
{fzjred},
{fzjgreen},
{fzjyellow},
{fzjlightblue},
}

\newcommand\globe{%
\hspace{0.1em}%
\tikz
\path [y=0.009ex,x=0.009ex,yscale=-1,inner sep=0pt,outer sep=0pt,fill]
    (90,90) circle (90)%
    (99,167) .. controls (128,164) and (155,142) ..
    (164,114) .. controls (166,107) and (170,96) ..
    (166,90) .. controls (160,91) and (159,109) ..
    (156,97) .. controls (155,87) and (142,82) ..
    (133,81) .. controls (131,89) and (146,83) ..
    (141,91) .. controls (136,97) and (125,103) ..
    (123,91) .. controls (123,87) and (115,80) ..
    (120,88) .. controls (121,94) and (124,107) ..
    (132,101) .. controls (137,110) and (120,115) ..
    (123,125) .. controls (119,133) and (112,141) ..
    (104,145) .. controls (92,148) and (91,132) ..
    (91,124) .. controls (91,115) and (87,106) ..
    (78,110) .. controls (72,112) and (67,111) ..
    (64,106) .. controls (60,103) and (58,99) ..
    (59,93) .. controls (58,83) and (67,77) ..
    (71,70) .. controls (63,68) and (69,57) ..
    (75,60) .. controls (70,51) and (83,49) ..
    (88,44) .. controls (95,46) and (108,42) ..
    (101,34) .. controls (97,26) and (100,47) ..
    (92,41) .. controls (88,39) and (80,37) ..
    (87,32) .. controls (93,26) and (103,29) ..
    (111,29) .. controls (118,28) and (132,30) ..
    (134,27) .. controls (127,19) and (116,16) ..
    (105,13) .. controls (91,10) and (76,11) ..
    (62,17) .. controls (55,20) and (43,22) ..
    (44,31) .. controls (47,38) and (37,50) ..
    (33,49) .. controls (32,43) and (28,38) ..
    (25,46) .. controls (18,54) and (31,66) ..
    (20,72) .. controls (10,78) and (10,91) ..
    (12,101) .. controls (14,112) and (26,115) ..
    (30,125) .. controls (37,130) and (47,134) ..
    (39,143) .. controls (32,149) and (47,154) ..
    (51,158) .. controls (65,165) and (83,170) ..
    (99,167) --
    cycle
    (124,138) .. controls (119,132) and (136,122) ..
    (129,133) .. controls (128,134) and (128,139) ..
    (124,138) --
    cycle
    (73,42) .. controls (63,39) and (81,27) ..
    (80,39) .. controls (80,41) and (75,44) ..
    (73,42) --
    cycle
    (100,77) .. controls (105,74) and (121,81) ..
    (115,72) .. controls (111,71) and (106,63) ..
    (103,69) .. controls (99,66) and (98,56) ..
    (93,60) .. controls (97,61) and (98,70) ..
    (93,64) .. controls (90,52) and (70,71) ..
    (81,70) .. controls (91,64) and (93,82) ..
    (100,77) --
    cycle
    (136,67) .. controls (138,62) and (133,52) ..
    (129,61) .. controls (131,63) and (133,77) ..
    (136,67) --
    cycle
    (123,63) .. controls (122,55) and (104,57) ..
    (112,64) .. controls (116,65) and (120,66) ..
    (123,63) --
    cycle;%
\hspace{0.25em}%
}

\lstset{ language=C++,
         basicstyle=\footnotesize\ttfamily, % Standardschrift
         numbers=none,               % Ort der Zeilennummern
         numberstyle=\footnotesize,          % Stil der Zeilennummern
         %stepnumber=2,               % Abstand zwischen den Zeilennummern
         numbersep=5pt,              % Abstand der Nummern zum Text
         tabsize=2,                  % Groesse von Tabs
         extendedchars=true,         %
         breaklines=true,            % Zeilen werden Umgebrochen
         keywordstyle=\color{fzjblue},
         frame=,
         keywordstyle=[1]\textbf,    % Stil der Keywords
         keywordstyle=[2]\textbf,    %
         keywordstyle=[3]\textbf,    %
         keywordstyle=[4]\textbf,   %\sqrt{\sqrt{}} %
         rulesepcolor=\color{fzjblue},
         fillcolor=\color{fzjblue},
         stringstyle=\color{red!50!brown}\ttfamily, % Farbe der String
         showspaces=false,           % Leerzeichen anzeigen ?
         showtabs=false,             % Tabs anzeigen ?
         xleftmargin=15pt,
         framexleftmargin=1pt,
         framexrightmargin=1pt,
         framexbottommargin=1pt,
         %backgroundcolor=\color{fzjblue},
         showstringspaces=false,      % Leerzeichen in Strings anzeigen ?
         nolol=true,
 }

\lstdefinelanguage
   {x64-Assembler}     % add a "x64" dialect of Assembler
   [x86masm]{Assembler} % based on the "x86masm" dialect
   % with these extra keywords:
   {morekeywords={vsubpd,vmulpd,vaddpd,vsqrtpd,vdivpd,
                  vsubps,vmulps,vaddps,vsqrtps,vdivps,
                  vmovss,vsubss,vmulss,vaddss,vsqrtss,vdivss,
                  vmovsd,vsubsd,vmulsd,vaddsd,vsqrtsd,vdivsd,
                  vmovaps,vmovapd,
                  vfmadd231ps,vfmadd231pd,
                  xmm,ymm,
                  CDQE,CQO,CMPSQ,CMPXCHG16B,JRCXZ,LODSQ,MOVSXD, %
                  POPFQ,PUSHFQ,SCASQ,STOSQ,IRETQ,RDTSCP,SWAPGS, %
                  rax,rdx,rcx,rbx,rsi,rdi,rsp,rbp, %
                  r8,r8d,r8w,r8b,r9,r9d,r9w,r9b}}
 \lstloadlanguages{% Check Dokumentation for further languages ...
         x64-Assembler,
         C,
         C++,
 }
\lstset{language=C++,
    keywordstyle=\color{fzjblue}\bfseries,
    commentstyle=\color{green!60!black},
    stringstyle=\ttfamily\color{red!50!brown},
    morekeywords={Real,Scalar}
    }
\lstset{literate=%
   *{0}{{{\color{red!20!violet}0}}}1
    {1}{{{\color{red!20!violet}1}}}1
    {2}{{{\color{red!20!violet}2}}}1
    {3}{{{\color{red!20!violet}3}}}1
    {4}{{{\color{red!20!violet}4}}}1
    {5}{{{\color{red!20!violet}5}}}1
    {6}{{{\color{red!20!violet}6}}}1
    {7}{{{\color{red!20!violet}7}}}1
    {8}{{{\color{red!20!violet}8}}}1
    {9}{{{\color{red!20!violet}9}}}1
}

%\DeclareCaptionFont{fzjblue}{\color{fzjblue}}
%\DeclareCaptionFont{fzjgray}{\color{fzjgray}}
%%  \captionsetup[lstlisting]{singlelinecheck=false, labelfont={fzjblue}, textfont={fzjblue}}


%\DeclareCaptionFont{white}{\color{white}}
%\DeclareCaptionFormat{listing}{\colorbox[cmyk]{0.09, 0.05, 0.05,0.01}{\parbox[b][1.5ex][c]{\textwidth}{\hspace{15pt}#1#2#3}}}
%%\captionsetup[lstlisting]{format=listing,labelfont=fzjblue,textfont=fzjgray, singlelinecheck=false, margin=0pt, font={sc, bf,footnotesize}}
%\captionsetup{labelformat=empty,labelsep=none}

% settings for the design of tables
\arrayrulecolor{fzjblue}

\part{How To Do Bibliographies}
\begin{frame}
    \frametitle{How to Do References}
    \begin{itemize}
      \item My talk is interesting \cite{ref1}.
      \item But this paper shows how it is really done \cite{ref2}.
      \item Wow, these guys did stupid things \cite{ref3}.
      \item Cite figures in the caption below, or in the text aside.
      \item Put your references in \texttt{references.bib}
    \end{itemize}
\end{frame}

\section{References}
\begin{frame}[allowframebreaks]
\frametitle{References}
    \tiny{\bibliographystyle{alpha}}
    %\tiny{\bibliographystyle{unsrt}}
    \bibliography{references}
\end{frame}



% add required configuration inside the config.local.tex
% please put additional configuration here, if needed


% ensure that separately built slides don't have any footer, yet
\iftoggle{prebuiltslides}{\setbeamertemplate{footer element1}{}
\setbeamertemplate{footer element2}{}
\setbeamertemplate{footer element3}{}
\setbeamertemplate{footer element4}{}
}{}


\setmathfont{Latin Modern Math}

\begin{document}

% \verb listings \tt ttfamily

% Measure size
% \begin{tikzpicture}
% \foreach \x in {-7,...,7}
% {
%   \foreach \y in {-2...,2}
%   {
%     \node (0) at (\x,\y) {\x,\y};
%   }
% }
% \end{tikzpicture}

% Hello world!
% \begin{frame}
% \frametitle{Hello world! t.miethlinger}
% \end{frame}

\includetitle{Slides/title}

\part{Introduction}
\makepart

\begin{frame}
\frametitle{About me}
\framesubtitle{(Thomas Miethlinger)}
\begin{itemize}
  \item Study: Master Physics
  \item Johannes Kepler University of Linz
  \item Institute for Theoretical Physics \\ Department Many Particle Systems
  \item Research:
    \begin{itemize}
    \item Quantum fluids
    \item Complex fluids
    \item Non-equilibrium statistical mechanics
    \end{itemize}
\end{itemize}
\end{frame}

\begin{frame}
\frametitle{About the GSP}

\begin{itemize}
  \item Supervisor: Dr. Edoardo Di Napoli
  \item Co-Supervisor: Dr. Xinzhe Wu
  \item SimLab Quantum Materials
  \item Research:
    \begin{itemize}
    \item Development and maintenance of numerical libraries
    \item Design and implementation of high-performance algorithms
    \item Development of new mathematical and computational models within a methodological framework
    \end{itemize}
    in the scope of computational materials science and quantum materials.
\end{itemize}
\end{frame}

\part{Introduction to HPX}
\makepart

\begin{frame}
\frametitle{Current sitution in high performance computing (HPC)}
Currently, speed-up in computing does not stem from higher CPU frequency, but increased parallelism.
However, we already face the following challenges in HPC:
\begin{itemize}
  \item Ease of programming
  \item Inability to handle dynamically changing workloads
  \item Scalability
  \item Efficient utilization of system resources
\end{itemize}
\(\implies\) a need for a new execution model: ParalleX, which is implemented by HPX
\end{frame}

\begin{frame}
\frametitle{ParalleX}
ParalleX is a new parallel execution model that offers an alternative to the conventional computation models(e.g. message passing):
\begin{itemize}
  \item Split-phase transaction model
  \item Message-driven
  \item Distributed shared memory
  \item Multi-threaded
  \item Futures synchronization
  \item Local Control Objects (LCOs)
  \item ...
\end{itemize}
ParalleX focusses on latency hiding instead of latency avoidance.
\end{frame}

\begin{frame}
\frametitle{About HPX}
\begin{itemize}
  \item High Performance ParalleX (HPX) is the first runtime system implementation of the ParalleX execution model.
  \item Development: STE||AR group \\ Louisiana State University \\ LSU Center for Computation and Technology
  \item Released as open source under the Boost Software License
  \item Aims to be a \textbf{C++ standards conforming implementation} of the Parallelism and Concurrency proposals for C++ 17/20/23/...
  \item This means: HPX is a C++ library that supports \textbf{dynamic adaptive resource management} and \textbf{lightweight task programming and scheduling} within the context of a \textbf{global address space}. 
\end{itemize}
\end{frame}

\begin{frame}
\frametitle{Comparison of HPX and OpenMP}
\begin{center}
\begin{tabular}{ |c|c| } 
 \hline
 HPX & OpenMP \\
 \hline
 C++ library & Compiler extension to C and Fortran \\
 Core language: \texttt{hpx::C++} & \texttt{\#pragma omp} directives \\
 Task-based parallelism & Parallel regions (fork-join model) \\
 AGAS (active global address space) & shared memory \\
 \hline
\end{tabular}
\end{center}
\vspace{0.3cm}
\begin{center}
\begin{tikzpicture}[
initnode/.style={circle, draw=green!60, fill=green!5, very thick, minimum size=5mm},
masternode/.style={rectangle, draw=red!60, fill=red!5, very thick, minimum size=5mm},
slavenode/.style={rectangle, draw=blue!60, fill=blue!5, very thick, minimum size=5mm},
]
\node[initnode]      at (-5, -1) {S};
\node[slavenode]     at (-4,  1) {2};
\node[slavenode]     at (-4,  0) {1};
\node[slavenode]     at (-4, -1) {0};
\node[slavenode]     at (-3,  1) {2};
\node[slavenode]     at (-3,  0) {1};
\node[slavenode]     at (-3, -1) {0};
\node[slavenode]     at (-2,  1) {2};
\node[slavenode]     at (-2,  0) {1};
\node[slavenode]     at (-2, -1) {0};
\node[initnode]      at (-1, -1) {E};
\draw[->] (-5+0.31,  -1) -- (-4-0.27, -1);
\draw[->] (-4.5,  1) -- (-4-0.27,  1);
\draw[->] (-4.5,  0) -- (-4-0.27,  0);
\draw[->] (-4.5, -1) -- (-4-0.27, -1);
\draw[-]  (-4.5, -1) -- (-4.5, 1);
\draw[->] (-4+0.27,  1) -- (-3-0.27,  1);
\draw[->] (-4+0.27,  0) -- (-3-0.27,  0);
\draw[->] (-4+0.27, -1) -- (-3-0.27, -1);
\draw[->] (-3+0.27,  1) -- (-2-0.27,  1);
\draw[->] (-3+0.27,  0) -- (-2-0.27,  0);
\draw[->] (-3+0.27, -1) -- (-2-0.27, -1);
\draw[->] (-2+0.27,  1) -- (-1.5,  1);
\draw[->] (-2+0.27,  0) -- (-1.5,  0);
\draw[-]  (-1.5, -1) -- (-1.5, 1);
\draw[->] (-2+0.27,  -1) -- (-1-0.31, -1);

\node[initnode]      at (1, -1) {S};
\node[masternode]    at (2, -1) {0};
\node[masternode]    at (3, -1) {0};
\node[slavenode]     at (3, -0) {1};
\node[slavenode]     at (3,  1) {2};
\node[masternode]    at (4, -1) {0};
\node[initnode]      at (5, -1) {E};
\draw[->] (1+0.31, -1) -- (2-0.27, -1);
\draw[->] (2+0.27, -1) -- (3-0.27, -1);
\draw[->] (2.5,  0) -- (3-0.27,  0);
\draw[->] (2.5,  1) -- (3-0.27,  1);
\draw[-]  (2.5, -1) -- (2.5, 1);
\draw[->] (3+0.27, -1) -- (4-0.27, -1);
\draw[->] (3+0.27,  0) -- (3.5,  0);
\draw[->] (3+0.27,  1) -- (3.5,  1);
\draw[-]  (3.5, -1) -- (3.5, 1);
\draw[->] (4+0.27, -1) -- (5-0.31, -1);
\end{tikzpicture}
\end{center}
\end{frame}

\begin{frame}
\frametitle{Comparison of HPX and MPI}
\begin{center}
\begin{tabular}{ |c|c| } 
 \hline
 HPX & MPI \\
 \hline
 C++ library & Interface specification for C and Fortran \\
 Core language: \texttt{hpx::C++} & Core language: MPI\_C, MPI\_F08 \\
 Task-based parallelism & Single program, multiple data (SPMD) \\
 AGAS (active global address space) & Explicit message passing \\
 \hline
\end{tabular}
\end{center}
\vspace{0.3cm}
\begin{center}
\begin{tikzpicture}[
initnode/.style={circle, draw=green!60, fill=green!5, very thick, minimum size=5mm},
masternode/.style={rectangle, draw=red!60, fill=red!5, very thick, minimum size=5mm},
slavenode/.style={rectangle, draw=blue!60, fill=blue!5, very thick, minimum size=5mm},
]
\node[initnode]      at (-5, -1) {S};
\node[slavenode]     at (-4,  1) {2};
\node[slavenode]     at (-4,  0) {1};
\node[slavenode]     at (-4, -1) {0};
\node[slavenode]     at (-3,  1) {2};
\node[slavenode]     at (-3,  0) {1};
\node[slavenode]     at (-3, -1) {0};
\node[slavenode]     at (-2,  1) {2};
\node[slavenode]     at (-2,  0) {1};
\node[slavenode]     at (-2, -1) {0};
\node[initnode]      at (-1, -1) {E};
\draw[->] (-5+0.31,  -1) -- (-4-0.27, -1);
\draw[->] (-4.5,  1) -- (-4-0.27,  1);
\draw[->] (-4.5,  0) -- (-4-0.27,  0);
\draw[->] (-4.5, -1) -- (-4-0.27, -1);
\draw[-]  (-4.5, -1) -- (-4.5, 1);
\draw[->] (-4+0.27,  1) -- (-3-0.27,  1);
\draw[->] (-4+0.27,  0) -- (-3-0.27,  0);
\draw[->] (-4+0.27, -1) -- (-3-0.27, -1);
\draw[->] (-3+0.27,  1) -- (-2-0.27,  1);
\draw[->] (-3+0.27,  0) -- (-2-0.27,  0);
\draw[->] (-3+0.27, -1) -- (-2-0.27, -1);
\draw[->] (-2+0.27,  1) -- (-1.5,  1);
\draw[->] (-2+0.27,  0) -- (-1.5,  0);
\draw[-]  (-1.5, -1) -- (-1.5, 1);
\draw[->] (-2+0.27,  -1) -- (-1-0.31, -1);

\node[initnode]      at (1, -1) {S};
\node[slavenode]     at (2,  1) {2};
\node[slavenode]     at (2,  0) {1};
\node[slavenode]     at (2, -1) {0};
\node[slavenode]     at (3,  1) {2};
\node[slavenode]     at (3,  0) {1};
\node[slavenode]     at (3, -1) {0};
\node[slavenode]     at (4,  1) {2};
\node[slavenode]     at (4,  0) {1};
\node[slavenode]     at (4, -1) {0};
\node[initnode]      at (5, -1) {E};
\draw[->] (1+0.31,  -1) -- (2-0.27, -1);
\draw[->] (1.5,  1) -- (2-0.27,  1);
\draw[->] (1.5,  0) -- (2-0.27,  0);
\draw[->] (1.5, -1) -- (2-0.27, -1);
\draw[-]  (1.5, -1) -- (1.5, 1);
\draw[->] (2+0.27,  1) -- (3-0.27,  1);
\draw[->] (2+0.27,  0) -- (3-0.27,  0);
\draw[->] (2+0.27, -1) -- (3-0.27, -1);
\draw[dashed] (2+0.27,  1) -- (3-0.27,  1);
\draw[dashed] (2+0.27,  1) -- (3-0.27,  0);
\draw[dashed] (2+0.27,  1) -- (3-0.27, -1);
\draw[dashed] (2+0.27,  0) -- (3-0.27,  1);
\draw[dashed] (2+0.27,  0) -- (3-0.27,  0);
\draw[dashed] (2+0.27,  0) -- (3-0.27, -1);
\draw[dashed] (2+0.27, -1) -- (3-0.27,  1);
\draw[dashed] (2+0.27, -1) -- (3-0.27,  0);
\draw[dashed] (2+0.27, -1) -- (3-0.27, -1);
\draw[->] (3+0.27,  1) -- (4-0.27,  1);
\draw[->] (3+0.27,  0) -- (4-0.27,  0);
\draw[->] (3+0.27, -1) -- (4-0.27, -1);
\draw[dashed] (3+0.27,  1) -- (4-0.27,  1);
\draw[dashed] (3+0.27,  1) -- (4-0.27,  0);
\draw[dashed] (3+0.27,  1) -- (4-0.27, -1);
\draw[dashed] (3+0.27,  0) -- (4-0.27,  1);
\draw[dashed] (3+0.27,  0) -- (4-0.27,  0);
\draw[dashed] (3+0.27,  0) -- (4-0.27, -1);
\draw[dashed] (3+0.27, -1) -- (4-0.27,  1);
\draw[dashed] (3+0.27, -1) -- (4-0.27,  0);
\draw[dashed] (3+0.27, -1) -- (4-0.27, -1);
\draw[->] (4+0.27,  1) -- (4.5,  1);
\draw[->] (4+0.27,  0) -- (4.5,  0);
\draw[-]  (4.5, -1) -- (4.5, 1);
\draw[->] (4+0.27,  -1) -- (5-0.31, -1);
\end{tikzpicture}
\end{center}
\end{frame}

\part{Overview of numerical linear algebra and its applications}
\makepart

\part{GEMM}
\makepart


\part{QR}
\makepart

% your talk should never end with an empty page.
% to make sure we don't see a black page, we just take the title page again
% in an ideal world, you will never "see" this frame
\includetitle{Slides/title}

\end{document}
